%% packages
\documentclass{article}
\usepackage[a4paper, left=2.0cm, right=2.0cm, top=3.5cm]{geometry}
\usepackage[ngerman]{babel}
\usepackage{graphicx}
\usepackage{multicol}
\usepackage{amssymb}
\usepackage{titlesec}
\usepackage{wrapfig}
\usepackage{blindtext}
\usepackage{lipsum}
\usepackage{caption}
\usepackage{listings}
\usepackage{fancyhdr}
\usepackage{nopageno}
\usepackage{float}
\usepackage{authblk}
\usepackage{mathtools} % e.g. alignment within matrix
\usepackage{bm} % provides shorthand for bold in math mode
\usepackage{dsfont} % \mathds makes double stroke digits
\usepackage{esdiff} % provides \diff
\usepackage[ISO]{diffcoeff}
\usepackage{xcolor}
\usepackage{csquotes} % e.g. provides \enquote
\usepackage[separate-uncertainty=true]{siunitx} % units
\usepackage{xcolor} % colored text
\usepackage[l3]{csvsimple}
\usepackage{subcaption}
\usepackage{physics}
\usepackage{hyperref}
\usepackage{nameref}
%\hypersetup{colorlinks=true, linkcolor=black, pdfhighlight={/N}}
\usepackage{tcolorbox}
\usepackage{amsthm}
\usepackage{gensymb} % add \degree in math mode?
\usepackage{newunicodechar} % define custom unicode characters
\usepackage{booktabs}
\usepackage{amsmath, amssymb, amsthm} 
\usepackage{geometry}
\usepackage[utf8]{inputenc}
\usepackage{enumerate}
\usepackage[shortlabels]{enumitem}

\usepackage{../pakete}
\usepackage{../aufgaben}

% \sisetup{
%   scientific-notation = auto,  % Automatically use scientific notation for large/small numbers
%   output-exponent-marker = \text{e}  % (optional) for formatting the exponent symbol
% }


%\fancyhf[]{}

%% custom stuff
% own units
\DeclareSIUnit \VSS {\ensuremath{V_\mathrm{SS}}}
\DeclareSIUnit \VS {\ensuremath{V_\mathrm{S}}}
\DeclareSIUnit \Veff {\ensuremath{V_\mathrm{eff}}}
\DeclareSIUnit \Vpp {\ensuremath{V_\mathrm{pp}}}
\DeclareSIUnit \Vp {\ensuremath{V_\mathrm{p}}}
\DeclareSIUnit \VRMS {\ensuremath{V_\mathrm{RMS}}}
\DeclareSIUnit \ASS {\ensuremath{A_\mathrm{SS}}}
\DeclareSIUnit \AS {\ensuremath{A_\mathrm{S}}}
\DeclareSIUnit \Aeff {\ensuremath{A_\mathrm{eff}}}
\DeclareSIUnit \App {\ensuremath{A_\mathrm{pp}}}
\DeclareSIUnit \Ap {\ensuremath{A_\mathrm{p}}}
\DeclareSIUnit \ARMS {\ensuremath{A_\mathrm{RMS}}}

% change subsection numbering to capital letters
\newcommand{\subsectionAlph}{ \renewcommand{\thesubsection}{\arabic{section}.\Alph{subsection}} }
% change subsection numbering to lowercase letters
\newcommand{\subsectionalph}{ \renewcommand{\thesubsection}{\arabic{section}.\alph{subsection}} }
% change subsubsection numbering to lowercase letters
\newcommand{\subsubsectionalph}{ \renewcommand{\thesubsubsection}{\arabic{section}.\arabic{subsection}.\alph{subsubsection}} }
% own fig. that works with multicols
\newenvironment{Figure}
  {\par\medskip\noindent\minipage{\linewidth}}
  {\endminipage\par\medskip}
\newcommand*{\inputPath}{./plot} % prepend this command to the argument of all input commands
\graphicspath{ {./images/}{./figure/}{../plot/}{../../plot/}{../../latex/assets/}{./assets/} }
% own enviroment for definitions
\newenvironment{definition}[1]
{\begin{quote} \noindent \textbf{\textit{#1\ifx&#1& \else : \fi}} \itshape}
{\end{quote}}

%\newunicodechar{°}{\degree}


% own commands
% \newcommand{\rarr}{$\to\,$} %A$\,\to\,$B
\newcommand{\defc}{black}
\newcommand{\colorT}[2][blue]{\color{#1}{#2}\color{\defc}}
\newcommand{\redq}{\color{red}(?)\color{\defc}}
\newcommand{\question}[1]{\colorT[purple]{\textbf{(#1)}}}
\newcommand{\todo}[1]{\colorT[red]{\textbf{(#1)}}}
\newcommand{\mr}{\mathrm}



%% preparation


%dachte wir können hier durchführung und gleich Auswertung der einzelnen Versuchsaufgaben schreiben und im Fazit ein gesamtfazit

%\displaystyle \lim_{x \to \infty}

\begin{document}
\begin{titlepage}
    \title{Elektronikpraktikum \\ Versuch 0: Einführung und Vorversuch}
    \author[1]{Carlos Pascua\thanks{s87cpasc@uni-bonn.de}}
    \author[1]{Anna Maróti\thanks{s32amaro@uni-bonn.de}}
    \author[1]{Cornelius Heiming\thanks{s64cheim@uni-bonn.de}}
    \affil[1]{Uni Bonn}
    %\date{\today}
\end{titlepage}
	\pagenumbering{gobble}
\maketitle
\tableofcontents
\newpage
\pagenumbering{arabic}

\pagestyle{fancy}
\fancyhead[R]{\thepage}
\fancyhead[L]{\leftmark}

\section{Theorie}

\subsection*{Einführung}
In diesem Versuch soll der Umgang mit  relevanten Geräten des Elektronikpraktikums geübt werden, indem verschiedene Signaltypen eines Funktionsgenerators analysiert werden. Im Anschluss wird die Anstiegszeit eines Rechtecksignals bestimmt.


\subsection*{Signalquellen}
\begin{itemize}
    \item Spitze-Spitze $U_{SS} [V_{SS}]$ oder $U_{pp} [V_{pp}]$: Die Differenz zwischen den niedrigsten und höchsten Spannungswert des Signals.
    \item Spitzenwert $U_{S} [V_{S}]$ oder $U_{p} [V_{p}]$: Maximalwert der auftretender Spannung 
    \item Effektivwert $U_{eff}$ : ist gleich der Spannung, die bei einer konstanten Gleichspannung und einem Ohmschen Widerstand die gleiche mittlere Leistung $P$ liefert. $U_{eff}= \sqrt{\langle U^2(t) \rangle}$
\end{itemize}


\subsection*{Das Oszilloskop}

Das Oszilloskop, ein elektronisches Messgerät mit welchem periodische Signale dargestellt werden können, besteht aus den folgenden Bauteilen
\begin{itemize}
    \item Netzteil
    \item Elektronenstrahlröhre: bestehend aus einem Glaskolben mit einer Glühkathode und einer Lochanode, die sich innerhalb eines Wehneltzylinders befinden (1). Zwischen Kathode, woraus Elektronen emittiert werden, und der Lochanode herrscht ein elektrsches Feld, wodurch die Elektonen zur Anode beschleunigt werden. Danach geht der gebündelte Elektronenstrahl durch zwei weiteren Elektroden mit elektrischen Feldern(2) wodurch er weiter abgelenkt wird. Schließlich trifft der Strahl (4) auf einen Leuchtschirm (3), auf dem er als leuchtender Punkt (5) sichtbar wird \ref{fig:elektronenstrahlröhre} \\
?? %ich muss die Einstellung raussuchen, wo ich hyperlink machen kann für die Bilder, damit da nicht so eine lonly 1 steht
    
    \item y-Verstärker: ein Breitbandverstärker
    
    \item Zeitablenkeinheit: Um ein stehendes Bild einer periodischen Spannung zu erhalten, wird mit einem Sägezahn-Signal am zweiten Elektrodenpaar eine Spannung angelegt. Hierdurch bewegt sich der Strahl mit konstanter Geschwindigkeit über den Leuchtschirm. Damit ein Bild entstehen kann muss noch die Bedingung erfüllt werden, dass die Frequenz der Messspannung ein ganzzahliges Vielfaches der Kippfrequenz ist.

    
\end{itemize}

\subsection*{Bandbreite}
Die Bandbreite wird bei diesem Versuch durch einen RC-Tiefpass begrenzt, welches die hohen Frequenzen vollständig dämpft.
\begin{align}
  B = f_{grenz}= \frac{1}{2\pi RC}=  \frac{1}{2\pi \tau}
\end{align}
Hier entspricht $R$ dem Ohmschen Widerstand und $C$ ist die Kapazität des Kondensators. 

\subsection*{Anstiegszeit}
Bei einem realen Rechtecksignal weist der Spannungsverlauf eine endliche Anstiegszeit $\Delta t$ auf, bevor das Signal seinen Maximalwert erreicht. Die Anstiegszeit ist definiert als der Zeitraum zwischen  $t(0,9\cdot U_{max}- 0,1\cdot U_{max})$, also die Zeit des Spannungsantiegs von 10$\%$ auf 90 $\%$.\autoref{fig:anstiegzeit} %schon wieder die Ref
Dieser gemessene Wert setzt sich aus dem theoretischen Wert der Anstiegszeit und dem Anstiegszeit des Oszilloskops folgendermaßen zusammen:
\begin{equation}
    \Delta t_{gemessen}^2 = \Delta t_{signal}^2 + t_{Oszi}^2
\end{equation}
wobei für den Oszilloskop gilt: $B \cdot \Delta t = 0,35$
\newpage 

\section{Voraufgaben}

\subsection*{Aufgabe A}
Folgende Größen gelten für die Spannung $U(t)= U_0 \cdot \sin{(\omega t)}$ :\\
\\
Spitze-Spitze-Spannung: \begin{equation}
        U_{SS}= 2 \cdot U_0
    \end{equation}
Spitzenspannung: 
    \begin{equation}
        U_{S}= U_0
    \end{equation}
 Effektivspannung:
    \begin{align*}
        U_{eff} &= \sqrt{ \langle U_0 ^2 \sin{(\omega t)}^2 \rangle}
        = \sqrt{ U_0 ^2 \langle \sin{(\omega t)}^2 \rangle} = \frac{U_0}{\sqrt{2}}
    \end{align*}
wobei $\langle \sin{(\omega t)}^2 \rangle = \frac{1}{2}$ ist.



\subsection*{Aufgabe B}
Der Effektivwert eines symmetrischen Rechtecksignals mit $U_S= 10 V$ entspricht:

\begin{equation}
    U_{eff}= \sqrt{\langle U_0 ^2 \rangle}= \sqrt{\langle U_S ^2 \rangle}= U_S = 10V
\end{equation}

\subsection*{Aufgabe C}
Für die Bestimmung des Innenwiderstandes des Generatorausgangs, kann die Spannung zunächst so ausgedrückt werden:
\begin{equation}
\label{Gl 1}
    U_n = U_0 \frac{R_n}{R_n + R_i}
\end{equation}
wobei $R_i$ der Innenwiderstand ist. Dieses kann zu folgender Formel umgestellt werden: 
\begin{equation*}
    U_0 = \frac{U_n \cdot (R_n +R_i)}{R_n}
\end{equation*}

Dann gilt die Formel für zwei unterschiedliche Widerstände:
\begin{equation*}
   U_1 \cdot \biggl(1 +\frac{R_i}{R_1}\biggr) = U_2 \cdot \biggl(1 +\frac{R_i}{R_2}\biggr) 
\end{equation*}
\begin{equation*}
    \Leftrightarrow U_1 + R_i \cdot \biggl(\frac{U_1}{R_1}\biggr) = U_2 + R_i \cdot \biggl(\frac{U_2}{R_2}\biggr)
\end{equation*}
Mit der  $I= \frac{U}{R}$ folgt: 
\begin{equation*}
    \Leftrightarrow U_1 +I_1 \cdot R_i =  U_2 +I_2 \cdot R_i
\end{equation*}
\begin{equation*}
    \Leftrightarrow R_i \cdot (I_1-I_2) = U_2 -U_1
\end{equation*}
\begin{equation}
\label{Gl 6}
    \Leftrightarrow R_i = \frac{U_2 - U_1}{I_1 - I_2}
\end{equation}
Anhand dieser Formel und den folgenden Angaben aus dem Skript kann der Wert des Innenwiderstands berechnet werden: \\
Maximalamplitude ohne Belastung: $U_\mathrm{1} = 20V_\mathrm{SS}$. \\
Maximalamplitude bei einer Belastung von $R = 50 \Omega$: $U_\mathrm{2} = 10V_\mathrm{SS}$. \\
Der Spitze-Spitze Wert für den Strom ist: $I_1 = 0A_{SS}$ und $I_2 = 0,2 A_{SS}$. Jetzt kann man \autoref{Gl 6} benutzen.
\begin{equation*}
    R_i = \frac{20-10}{0,2-0} = 50 \Omega
\end{equation*}



%!!!!!!!!!!Woher die werte? sind sie noch gleich?

\subsection*{Aufgabe D}
Anordnung des Oszilloskops wurde studiert und sich über die Funktion der Elemente aufgeklärt. 

\subsection*{Aufgabe E}
\label{subsec:afgE}
Für den Tiefpass soll folgender Zusammenhang geprüft werden:  $B\cdot \Delta t = 0,35$.\\
Bei Tiefpassfilter gilt für die Bandbreite: 
\begin{equation}
    B = f_{grenz} = \frac{1}{2\pi RC} = \frac{1}{2\pi \tau}
\end{equation}
Mit der Anstiegszeit, welches die Zeit ist, die ein Signal braucht um von 10\% auf 90\% anzusteigen, ergibt sich die Formel zu: \\
 \begin{equation*}
     \Delta t = (ln(0,9)-ln(0,1))\cdot \tau 
 \end{equation*}
 \begin{equation*}
     \iff B = \frac{1}{2\pi \tau} \cdot (ln(0,9)-ln(0,1))\cdot \tau 
 \end{equation*}
 \begin{equation*}
     \iff B = \frac{ln(0,9)-ln(0,1)}{2 \pi} \approx 0,35
 \end{equation*}
 Somit wurde der Zusammenhang bewiesen. 
 
\newpage
		%TODO Voraufgaben
% === Aufbau, Durchführung, Messwerte und Auswertung ===
	\section{Versuchsaufbau, -durchführung, Messwerte und Auswertung}
		\subsection*{Bestimmung der Anstiegszeit des Oszillographen}
			\begin{figure}[H]
				\centering
				\includegraphics[width=0.8\textwidth]{figs/Aufbau_0_1_Oszilloskop.png}
				\caption{Schaltplan Funktionsgenerator und Oszilloskop~\cite{anleitung}}
				\label{fig:aufbau_0_1_oszilloskop.png}
			\end{figure}
			\begin{enumerate}[(a)]
				\item In dieser Aufgabe soll der Oszillograph untersucht werden. Dafür wird der Generatorausgang mit dem CH1 Eingang des Oszilloskops mittels eines Koaxialkabels verbunden. Nach dem Triggern können verschiedene Oszillogramme beobachtet werden.
				\item Danach wird ein Rechtecksignal mit einer Frequenz von $\SI{2}{\mega\hertz}$ und einer Amplitude von beispielsweise $\SI{1}{\volt}$ eingestellt. Die Anstiegszeit $\Delta t_\mathrm{gemessen}$ wird bestimmt, indem die Zeitdifferenz zwischen dem Zeitpunkt, an dem $U(t)$ den Wert $0.1 U_0$ erreicht, und dem Zeitpunkt, an dem $U(t)$ den Wert $0.9 U_0$ erreicht, gemessen wird.
				\item Zuletzt wird der Generatorausgang mit einem R-C-Filter verbunden. Es soll eine Zeitkonstante von $\tau = \mathcal{O}(\SI{10}{\micro\second} \text{ bis } \SI{100}{\micro\second})$ realisiert werden, also können bspw. $R = \SI{1}{\kilo\ohm}$ und $C = \SI{10}{\micro\farad}$ gewählt werden. Der R-C-Filter wird erneut mit dem CH1-Eingang des Oszilloskops verbunden. Der Generator wird auf ein Sinus-Signal mit 10 verschiedenen Frequenzen eingestellt. Dabei wird sich jeweils die Amplitude am Signalgenerator notiert.
			\end{enumerate}
			\MesswUndAusw
	
				\begin{figure}[H]
					\centering
					\includegraphics[width=0.8\textwidth]{MesswerteVersuch0/A0000DS.png}
					\includegraphics[width=0.8\textwidth]{MesswerteVersuch0/A0001DS.png}
					\caption{Oben: Rechtecksignal, unten: Dreieckssignal}
					\label{fig:A0000DS,1}
				\end{figure}
				\begin{figure}[H]
					\centering
					\includegraphics[width=0.8\textwidth]{MesswerteVersuch0/A0002DS.png}
					\caption{Sinussignal}
					\label{fig:A0002DS}
				\end{figure}
	
				\begin{figure}[H]
					\centering
					\includegraphics[width=0.8\textwidth]{MesswerteVersuch0/A0005DS.png}
					\caption{Rechteckssignalanstieg (detailliert)}
					\label{fig:A0005DS}
				\end{figure}
				Aus der Grafik \ref{fig:A0005DS} kann die Anstiegszeit $\Delta t_\mathrm{gemessen}$ abgelesen werden. Dafür wurden die $10\%$ und $90\%$ markiert und damit die Zeitpunkte ermittelt, an denen die Spannung $U(t)$ den Wert $0.1 U_0$ und $0.9 U_0$ erreicht. Die Anstiegszeit $\Delta t_\mathrm{gemessen}$ beträgt dann $\SI{16}{\nano\second}$ (Genau $8$ Einheiten der Skala, die je $\frac{1}{5}$ von $\SI{10}{\nano\second}$ sind). Der Messfehler beträgt $4$ Pixel, also $\Delta(\Delta t_\mathrm{gemessen}) =  \SI{1}{\nano\second}$. Der Betriebsanleitung des Oszilloskops \cite{oszibedienungsanleitung} zufolge beträgt die Bandbreite des Oszilloskops $B = \SI{50}{\mega\hertz}$. Daraus folgt mit der Formel $B \cdot \Delta t = \SI{0.35}{}$ (siehe Voraufgabe~\ref{voraufgabeD}) die Anstiegszeit:
				\begin{align*}
					\Delta t_\mathrm{Oszi} = \frac{0.35}{\SI{50}{\mega\hertz}} = \SI{7}{\nano\second}
				\end{align*}
				Für die Anstiegszeit des Signals $\Delta t_\mathrm{Signal}$ gilt nach der Anleitung\cite{anleitung} die Näherung:
				\begin{align*}
					(\Delta t_\mathrm{Signal})^2 &= (\Delta t_\mathrm{Oszi})^2 - (\Delta t_\mathrm{gemessen})^2 \\
					\Delta t_\mathrm{Signal} &= \sqrt{(\Delta t_\mathrm{Oszi})^2 - (\Delta t_\mathrm{gemessen})^2} \\
					&= \sqrt{(\SI{16}{\nano\second})^2 - (\SI{7}{\nano\second})^2} \approx \SI{14.39}{\nano\second}
				\end{align*}
				Mittels Gaußscher Fehlerfortpflanzung ergibt sich:
				\begin{align*}
					\Delta(\Delta t_\mathrm{Signal}) &= \frac{2 t_\mathrm{gemessen}}{\sqrt{(\Delta t_\mathrm{gemessen})^2 - (\Delta t_\mathrm{Oszi})^2}} \cdot \Delta(\Delta t_\mathrm{gemessen}) \\
					&= \frac{2 \SI{16}{\nano\second}}{\sqrt{(\SI{16}{\nano\second})^2 - (\SI{7}{\nano\second})^2}} \cdot \SI{1}{\nano\second} \\
					&\approx \SI{2.3}{\nano\second}
				\end{align*}
				Also ergibt sich für die Anstiegszeit des Signals:
				\begin{align*}
					\Delta t_\mathrm{Signal} &= \SI{14.4}{\nano\second} \pm \SI{2.3}{\nano\second}
				\end{align*}
	
				\begin{figure}[H]
					\centering
					\includegraphics[width=0.8\textwidth]{pythonAuswertungen/Versuch 0 Aufgabe 1 (c)_plot.jpg}
					\caption{Dämpfungsplot}
					\label{0_1_(c)_Dämpfung}
				\end{figure}
				%TODO: c) Tabelle machen, auswerten.
		

			%TODO Aufbau/Schaltskizze (beschrieben in wenigen Worten)
			%TODO Versuchsdurchführung: Messgrößen, unabh. Parameter, Messmethode, Einheiten, Genauigkeit, wie oft
			%TODO Abgezeichnetes Messprotokoll vorhanden?
			%TODO Auswertung (in Worten & Formeln)
			%TODO Formeln symbolisch und numerisch
			%TODO Runden
			%TODO Grafiken & Diagramme: Überschrift, Messwerte mit Fehlerbalken, (nur) gefittete Kurven, Achsenbeschriftung
			%TODO Fehlerrechnung und -diskussion
			%TODO Jede Tabelle eine Formel
		\section{Endresultat}
			%TODO Vollständiger Satz
		\section{Ergebnisdiskussion/Plausibilitätskontrolle}
			
			%TODO Messergebnisse bewerten und evtl. mit Literaturwerten vergleichen
			%TODO Untersuchung Fehlerquellen (statistisch, systematisch, blödsinnig)
		%% \printbibliography
\end{document}
