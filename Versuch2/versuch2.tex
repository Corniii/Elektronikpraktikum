% !TeX spellcheck = de_DE
\documentclass{article} 
%\KOMAoptions{fontsize=12pt, paper=a4}
%\KOMAoptions{DIV=11}
\usepackage[ngerman]{babel}
\def\nummer{1}
\def\datum{29.04.2025}
\def\titel{Ausbreitung von Signalen auf Leitungen}
%TODO: Name, Nummer, Datum	
\title{Versuch \nummer~-  \titel}
\date{\datum}
\usepackage{amsmath, amssymb, amsthm} 
\usepackage{geometry}
\usepackage[utf8]{inputenc}
\usepackage{enumerate}
\usepackage[shortlabels]{enumitem}
\usepackage{authblk}
\usepackage{fancyhdr}
\usepackage{titlesec}

\usepackage{../pakete}
\usepackage{../aufgaben}


\addbibresource{../referenzen.bib}
\geometry{a4paper, left=3cm, right=3cm, top=3cm, bottom=3cm}

\newtheorem{theorem}{Satz}
\newtheorem{lemma}[theorem]{Lemma}
\newtheorem{korollar}{Korollar}[section]
\theoremstyle{definition}
\newtheorem{definition}[theorem]{Definition}
\newtheorem{beispiel}[theorem]{Beispiel}
\newtheorem{satz}[theorem]{Satz}
\usepackage{authblk}
\renewcommand\Authands{ und }

%\displaystyle \lim_{x \to \infty}



\begin{document}
    \title{Elektronikpraktikum \\ \textbf{Versuch 2: Diodenkennlinien}}
    \author[1]{Carlos Pascua \thanks{s87cpasc@uni-bonn.de}}
    \author[1]{Anna Maróti\thanks{s32amaro@uni-bonn.de}}
    \author[1]{Cornelius Heiming\thanks{s64cheim@uni-bonn.de}}
    \affil[1]{Uni Bonn}
    %\date{\today}
	\pagenumbering{gobble}
    \begin{titlepage}
     \maketitle   
    \end{titlepage}
        
\tableofcontents
\newpage
\pagenumbering{arabic}

\pagestyle{fancy}
\fancyhead[R]{\thepage}
\fancyhead[L]{\leftmark}

% === Einleitung ===
\section{Einleitung}

In diesem Versuch liegt der Fokus auf Dioden und ihren elektrischen Eigenschaften. 
Als erstes werden die Kennlinien verschiedener Dioden aufgezeichnet. 
Danach wird mit Hilfe eines Gleichrichters eine Wechselspannung in Gleichstrom umgewandelt, 
was dann unter der Verwendung eines Kondensators noch weiter geglättet wird. 
Und zuletzt wird analysiert, wie eine Zenerdiode die Spannung in einem Stromkreis stabilisieren kann.

% === Theorie ====
\section{Theorie}
    \subsection{Halbleiter und Dotierung}
    Halbleiter besitzen eine elektrische Leitfähigkeit, die zwischen der von Metallen und Isolatoren
    liegt. Sie kann gezielt durch Anlegen einer Spannung oder Erhöhung der Temperatur beeinflusst
    werden. Dieses lässt sich mit dem Bändermodell erklären. Das Bändermodell besagt, dass man
    Halbleiter als Valenzband und Leitungsband approximieren kann. Zwischen den beiden Gebieten
    existiert eine Bandlücke mit der Breite der materialabhängigen Gap-Energie. Damit also Strom
    fließen kann, müssen die Elektronen genug Energie haben, um diese Bandlücke zu passieren.
    Wie viele Ladungsträger vorhanden sind, kann durch gezielte Verunreinigung des Stoffes (sogennante Dotierung) explizit beeinflussen kann. 
    Konkret verwendet man häufig Kristalle von Elementen mit vier Valenzelektronen (bspw. Silizium oder Germanium), welche mit Elementen der 
    dritten bzw. fünften Hauptgruppe verunreinigt, d.h. dotiert werden. Ersteres liefert einen Mangel an freien Elektronen, um die vervollständigte 
    Kristallstruktur zu erreichen, weshalb diese als p(ositiv)-dotiert gelten. Andersherum sorgen Elemente der fünften Hauptgruppe für einen Elektronenüberschuss,
     also eine n(egativ)-Dotierung.

    Bringt man nun eine p-Dotierte und eine n-Dotierte Schicht zusammen, können Elektronen übergehen. Dadurch entsteht allerdings eine elektrische Kraft, welches der molekularen Bindungskraft entgegen wirkt, 
    es ergibt sich also ein Gleichgewicht von Kräften. In dem Bereich des Kräftegleichgewichts liegen keine freien Ladungsträger vor, da diese vollständig gebunden sind, 
    d.h. diese Schicht ist nichtleitend. Durch ein äußeres elektrisches Feld (Spannung) kann aber das innere elektrische Feld überlagert werden, was, je nach Richtung der 
    Spannung eine Vergrößerung bzw. Verkleinerung des elektrischen Felds zufolge hat. Ab dem Punkt, an welchem sich äußeres und inneres Feld ausgleichen, liegt keine 
    Grenzschicht mehr vor, die Kombination ist also leitend.  
    \subsection{Diode}
    Eine Diode ist ein elektronisches Bauelement, welches den Strom in genau einer Richtung durchlässt. 
    Man spricht von Durchlass- und Sperrrichtung Diese wird zumeist mit einem Halbleiter, wie oben beschrieben, umgesetzt.
    \begin{figure}[H]
        \centering
        \includegraphics[width=0.8\textwidth]{figs/fig2_1.png}
        \caption{Schaltzeichen Diode\cite{anleitung}}
        \label{fig:Abb2.1}
    \end{figure}

    \subsection{Realität \& Diodenkennlinien}
    In der Realität kann eine Diode die idealisierende Bedingung von Durchlass in einer Richtung und Sperrung 
    in der entgegengesetzten nicht erreichen. In Durchlassrichtung muss zunächst die Grenzschicht mittels eines 
    äußeren Felds überwunden werden, danach verhält sich die Diode wie ein Leiter. In der Sperrichtung kann zunächst stets 
    ein kleiner Strom beobachtet werden, da die freien Ladungsträger der n-dotierten Schicht sich an der Kathode befinden, also einfach 
    abfließen können. Die Elektronen fließen dann durch den Halbleiter nach. Wenn die Spannung in Sperrrichtung groß genug wird, durchbricht 
    die Diode, ist zerstört und leitet den Strom ebenfalls. Dadurch kann man die U-I-Beziehung an einer Diode betrachten - die Diodenkennlinie \ref{fig:Abb2.2}:
    \begin{figure}[H]
        \centering
        \includegraphics[width=0.8\textwidth]{figs/fig2_2.png}
        \caption{Diodenkennlinie\cite{anleitung}}
        \label{fig:Abb2.2}
    \end{figure}

    \subsection{Zenerdiode}

Die Zenerdiode ist eine spezielle Diode, die gezielt im Durchbruchbereich in Sperrrichtung betrieben werden kann. 
Hierdurch kann sie benutzt werden, um Spannungen einzugrenzen und zu stabilisieren. 

    \subsection{Gleichrichter}
    Die grundlegende Eigenschaft einer Diode kann verwendet werden, um eine Wechselspannung in eine Gleichspannung umzuwandeln. Am einfachsten geschieht dies durch in-Reihe-Schalten einer Diode, sodass nur Spannungen einer Richtung auftauchen (vgl. \ref{fig:Abb2.4mod} a)). Nachteil daran ist, dass dann mit der Frequenz der Wechselspannnung negative Spannungen, welche durch die Diode annuliert werden, vorliegen. Um dies zu vermeiden, wird ein Zweiweggleichrichter verwendet (vgl. \ref{fig:Abb2.4mod} b)).
    \begin{figure}[H]
        \centering
        \includegraphics[width=0.8\textwidth]{figs/fig2_4mod.png}
        \caption{Ein- und Zweiweggleichrichter\cite{anleitung}}
        \label{fig:Abb2.4mod}
    \end{figure}

    

    Auch wenn dies die Spannung gleichrichtet, kommen diverse realistische Effekte zu tragen, welche dafür sorgen, dass keine glatte Gleichspannung entsteht, diese Abweichung wird "Brumm" genannt. Um dies nun auch noch auszuglätten wird parallel zum Verbraucher ein Kondensator angelegt.
% === Voraufgaben ===
\section{Voraufgaben}

% == A ==
    \begin{voraufgabe}{Was bestimmt die Dicke der Grenzschicht bei einem p-n-Halbleiter?}
        Die Dicke der Grenzschicht ist erstmal materialabhängig. Trotzdem kann ihre relative Dicke durch die Dotierung und einem äußeren elektrischen Feld beeinflusst werden.
        Die Dotierung sorgt für die Bereitstellung von Ladungsträgern (Elektronen oder "Löcher"), welche die Grenzschicht bilden und durch das äußere elektrische Feld beeinflusst werden.
    \end{voraufgabe}
% == B ==
    \begin{voraufgabe}{Wie ändert sich die Kapazität einer Diode im Sperrfall mit der angelegten Spannung?}
        Im Sperrfall kann die Diode Modellhaft als Plattenkondensator mit dem Sperrband als Dielektrikum modelliert werden.
         Für die Dicke des Sperrbands $d$, dessen relative elektische Permeabilität $\epsilon_r$ und die Durchschnittsfläche 
         der Diode $A$ gilt:
        \begin{equation}
            C = \epsilon_0 \epsilon_r \frac{A}{d} \propto \frac{1}{\sqrt{U_\mathrm{ext}}}
            \label{Kapazität einer Diode}
        \end{equation}
    \end{voraufgabe}
    
% == C ==
\begin{voraufgabe}{Skizzieren Sie den Kennlinienverlauf, $I=f(U)$, der Zweipole aus Abb. 2.3 (\ref{fig:Abb2.3}) $(R = \SI{100}{\ohm}; D = \text{Diode})$. Erläutern Sie bei c) und d) den Einfluss der Widerstände.}
    \begin{figure}[H]
        \centering
        \includegraphics[width=0.8\textwidth]{figs/fig2_3.png}
        \caption{Einige Zweipole\cite{anleitung}}
        \label{fig:Abb2.3}
    \end{figure}
    \begin{figure}[H]
        \centering
        \includegraphics[width=0.8\textwidth]{figs/fig2_C.png}
        \caption{Kennlinienverläufe einiger Zweipole}
        \label{figC}
    \end{figure}
    Bei c) ist der Widerstand hinter der Diode in Reihe geschaltet. Für den Bereich, in dem die Diode keinen Strom durchlässt, ist der Strom $0$, woran auch der Widerstand nichts ändert. Lässt die Diode allerdings Strom durch, so spiegelt sich der lineare Anstieg durch den Widerstand wider.

    Bei d) sind Diode und Widerstand parallel geschaltet. In Sperrrichtung sorgt eine zu hohe Spannung für eine Zerstörung der Diode, wodurch diese zu einem Leiter wird, weshalb der Strom haupsächlich über diese fließt. Bei einer geringen Sperrrichtungsspannung wird die Diode nahezu ignoriert, weshalb ein linearer Anstieg vorliegt. Wird die Diode leitend, handelt es sich um parallelgeschaltete Widerstände, was einen höheren Strom ermöglicht.

\end{voraufgabe}
% == D ==
\begin{voraufgabe}{Skizzieren Sie den zeitlichen Verlauf der Ausgangsspannungen der Schaltungen in Abb. 2.4 (\ref{fig:Abb2.4}) (a) und (b), wenn die Eingangsspannung eine weit über der Durchlassspannung der Dioden liegende Sinusspannung ist.}
    \begin{figure}[H]
        \centering
        \includegraphics[width=0.8\textwidth]{figs/fig2_4.png}
        \caption{Ein- und Zweiweggleichrichter\cite{anleitung}}
        \label{fig:Abb2.4}
    \end{figure}
    \begin{figure}[H]
        \centering
        \includegraphics[width=0.8\textwidth]{figs/fig2_D.png}
        \caption{Zeitlicher Verlauf der Ausgansspannungen von Ein- und Zweiweggleichrichter}
        \label{figD}
    \end{figure}
\end{voraufgabe}
% == E ==
\begin{voraufgabe}{Wie muss $C$ dimensioniert sein, um die Welligkeit der Spannung über $R$ möglichst klein zu halten?}
Ein Glättungskondensator lädt sich durch die Spitzen des gepulsten Stroms und entlädt sich wenn
dieser nachlässt. Auf dieser Weise wird der Stromfluss geglättet und der Gleichstrom ist stabiler.
Die Welligkeit der Ausgangsspannung wird durch die Entladung des Kondensators bestimmt
Je größer die Kapazität $C$, desto länger dauert der Entladevorgang des Kondensators,
was eine längere Kompensation des Brummens und somit eine stärkere Stabilisierung der 
Ausgansspannung zufolge hat.

\end{voraufgabe}
% == F ==
\label{F}
\begin{voraufgabe}{Wie würden Sie Strom- und Spannungsmessgerät zur Messung der Kennlinie in Durchlassrichtung und in Sperrrichtung anordnen? Berückstichtigen Sie die Innenwiderstände der beiden Geräte.}
Im Allgemeinen müssen Strommessgeräte in Reihe und Spannungsmessgeräte parallelgeschaltet werden. 
In Sperrichtung hat die Diode einen vergleichsweise hohen Widerstand, welcher für einen geringen 
Strom durch die Diode sorgt. Dieser sollte möglichst genau gemessen werden, weshalb die Spannungsmessung 
um die Diode und das Strommessgerät herum erfolgen sollte. Andersherum hat die Diode in Durchlassrichtung 
einen geringen Widerstand, was für einen hohen Strom sorgt, die Abweichung durch eine Spannungsmessung direkt 
an der Diode sind also eher gering, also zu bevorzugen.


\end{voraufgabe}
% == G ==
\label{G}
\begin{voraufgabe}{Wie kann man sich eine zu einem Strom proportionale Spannung herstellen?}
    Über einen ohmschen Widerstand fällt die Spannung proportional zur Stromstärke ab.
     \[U = R \cdot I\]

\end{voraufgabe}
% == H ==

\begin{voraufgabe}{Für Abb. 2.8 (\ref{fig:Abb2.8}): Berechnen Sie größenordnungsmäßig die größte Kapazität, die benutzt werden darf, ohne die Grenzwerte der Si-Diode zu überschreiten. Nehmen Sie dazu an, dass sich $U$ beim Einschalten um $\SI{1}{\volt}$ in $\SI{100}{\micro\second}$ ändert und vernachlässigen Sie den Einfluss von $R_L$.}
    \begin{figure}[H]
        \centering
        \includegraphics[width=0.8\textwidth]{figs/fig2_8.png}
        \caption{Einweggleichrichtung\cite{anleitung}}
        \label{fig:Abb2.8}
    \end{figure}
Der Maximalstrom $I_\mathrm{max}$ einer Si-Diode beträgt $\SI{1000}{\milli\ampere}$ (Seite 27 der Anleitung\cite{anleitung}). Wegen Ladungserhaltung gilt:
\begin{equation*}
    C_\mathrm{max} \Delta U = I_\mathrm{max} \Delta t
\end{equation*}
Also:
\begin{equation*}
    C_\mathrm{max} = I_\mathrm{max} \frac{\Delta t}{\Delta U} = \SI{1}{\ampere} \frac{\SI{100}{\micro\second}}{\SI{1}{\volt}} = \SI{100}{\micro\farad}
\end{equation*}
\end{voraufgabe}
% == I ==
\begin{voraufgabe}{Skizzieren Sie den zeitlichen Verlauf der Spannung am Ausgang der Schaltungen in Abb. 2.9.}
    \begin{figure}[H]
        \centering
        \includegraphics[width=0.8\textwidth]{figs/fig2_9.png}
        \caption{Ein- und Zweiweggleichrichtung mit dem Diodenschaltbrett\cite{anleitung}}
        \label{fig:Abb2.9}
    \end{figure}
    \begin{figure}[H]
        \centering
        \includegraphics[width=0.8\textwidth]{figs/fig2_I.png}
        \caption{Zeitlicher Verlauf der Spannung am Ausgang der Ein- und Zweiweggleichrichtungsschaltungen}
        \label{figI}
    \end{figure}
\end{voraufgabe}
% == J ==
\begin{voraufgabe}{Skizzieren Sie die Lastabhängigkeit der Spannung $U'$ der Schaltung auf der linken Seite in Abb. 2.11. Geben Sie die Formel an, aus der sich $U'$ in Abhängigkeit von $U_0$,$R$ und $R_L$ berechnen lässt. Was sind die Extremwerte für $U'$ und $I$?}
    \begin{figure}[H]
        \centering
        \includegraphics[width=0.8\textwidth]{figs/fig2_11.png}
        \caption{Spannungsstabilisierung mittels Zenerdiode\cite{anleitung}}
        \label{fig:Abb2.11}
    \end{figure}
    \begin{figure}[H]
        \centering
        \includegraphics[width=0.8\textwidth]{figs/fig2_J.png}
        \caption{Lastabhängigkeit der Spannung $U'$}
        \label{figJ}
    \end{figure}
    Gemäß Kirchhoff gilt:
    \begin{align*}
        U_0 &= R I + R_L I\\
        U' &= I R_L
        \implies U' = U_0 \frac{R_L}{R+R_L}
    \end{align*}
    Da $R$ und $R_L$ nichtnegativ sind, gilt:
    \begin{align*}
        U_\mathrm{\max{}}' &= U_0 \\
        U_\mathrm{\min{}}' &= 0 \\
        I_\mathrm{\max{}}' &= \frac{U_0}{R} \\
        I_\mathrm{\min{}}' &= 0
    \end{align*}

\end{voraufgabe}
% == K ==
\begin{voraufgabe}{Innerhalb welches \underline{Wertebereiches} muss bei dieser Dimensionierung der Arbeitswiderstand $R$ liegen, damit die Ausgangsspannung $U'$ bei der Zenerspannung von $\SI{8.2}{\volt}$ stabilisiert wird?}
    Betriebsdaten der Zenerdiode:
    \begin{align*}
        U_\mathrm{Z}&=\SI{8.2}{\volt}\\
        U_\mathrm{0,max}&=\SI{22}{\volt}\\
        U_\mathrm{0,min}&=\SI{16}{\volt}\\
        I_\mathrm{Z,max}&=\SI{100}{\milli\ampere}\\
        I_\mathrm{Z,min}&=\SI{2}{\milli\ampere}
    \end{align*}
    Falls $R_\mathrm{L}=\infty$, läuft der gesamte Strom durch die Zenerdiode. Dieser darf $I_\mathrm{Z,max}$ nicht überschreiten. Daher muss $R_\mathrm{min}$ entsprechend gewählt werden:
    \begin{align*}
        I_\mathrm{R}&=I_\mathrm{Z,max}=\frac{U_\mathrm{0,max}-U_\mathrm{Z}}{R}\\
        \Rightarrow R&>\frac{U_\mathrm{0,max}-U_\mathrm{Z}}{I_\mathrm{Z,max}}=\SI{138}{\ohm}
    \end{align*}
    Weiterhin darf unter keinen Umständen der Wert für $I_\mathrm{Z,min}=\SI{2}{\milli\ampere}$ unterschritten werden. Hierfür muss der Fall $U_0=U_\mathrm{0,min}$ und $R_\mathrm{L}=\SI{200}{\ohm}$ betrachtet werden:
    \begin{align*}
        I_\mathrm{Z,min} &= I_\mathrm{R}-I_\mathrm{L} \\
        &= \frac{U_\mathrm{0,min}-U_\mathrm{Z}}{R}-\frac{U_\mathrm{Z}}{R_\mathrm{L,min}} \\
        \Rightarrow R &< \frac{U_\mathrm{0,min}-U_\mathrm{Z}}{I_\mathrm{Z,min}+\frac{U_\mathrm{Z}}{R_\mathrm{L,min}}} = \SI{181}{\ohm}
    \end{align*} 
    \label{VoraufgabeK}   
\end{voraufgabe}


\clearpage
\section{Versuchsaufbau, -durchführung, Messwerte und Auswertung}
% === Aufbau, Durchführung, Messwerte und Auswertung ===
% == Aufgabe 1 ==
\begin{aufgabe}{Statische Messung der Diodenkennlinie} 
    \aufbau
    \begin{figure}[H]
        \centering
        \includegraphics[width=0.8\textwidth]{figs/fig2_6.png}
        \caption{oberer Teil des Diodenschaltbretts\cite{anleitung}}
        \label{aufbau_2_1}
    \end{figure}
Zunächst soll die Diodenkennlinie der unbekannten Dioden D1 und D2 mittels einer statischen Messung ermittelt werden.
 Dafür wird entsprechend der Voraufgabe~F~(\ref{F}) mit Ampère- und Voltmeter gemessen. Die Messwerte werden dann in einer 
 Tabelle aufgelistet, wie zum Beispiel \ref{tab:D1duchlass}, \ref{tab:D1sperr}. 

\begin{table}[h!]
    \centering
    \begin{tabular}{|l|l|}
    \hline
    \textbf{Spannung $U$ in $[V]$} & \textbf{Strom $A$ in $[mA]$} \\
    \hline
    $0.104$ & $0$ \\
    $0.201$ & $0$\\
    $0.304$ &  $2 \cdot 10^-3$ \\
    $0.401$ & $2 \cdot 10^-3$\\
    $0.501$ & $0.13$ \\
    $0.601$ & $0.50$ \\
    $0.701$ & $1.1$ \\
    $0.802$ & $1.9$ \\
    $0.902$ & $2.7$ \\
    $1.003$ & $3.5$ \\
    $1.513$ & $8.5$ \\
    $2.061$ & $13.5$ \\
    $5.02$ & $41.8$ \\
    \hline
    \end{tabular}
    \caption{Kennlinie D1 in Durchlassrichtung}
    \label{tab:D1duchlass}
    \end{table}
    
    \newpage

    \begin{table}[h!]
    \centering
    \begin{tabular}{|l|l|}
    \hline
    \textbf{Spannung $U$ in $[V]$} & \textbf{Strom $A$ in $[\mu A]$} \\
    \hline
    $-1.004$ & $0$ \\
    $-2.009$ & $0$\\
    $-3.009$ & $0$\\
    $-4.002$ & $0$\\
    $-5.10$ & $0.5$\\
    $-6.04$ & $0.5$\\
    $-7.05$ & $0.8$\\
    $-8.07$ & $0.9$\\
    $-9.004$ & $1$\\
    $-10.05$ & $1$\\
    $-11.00$ & $1,1$\\
    $-12.03$ & $1.2$\\
    \hline
    \end{tabular}
    \caption{Kennlinie D1 in Sperrrichtung}
    \label{tab:D1sperr}
    \end{table}

    Nun bringt man die beiden Tabellen zusammen und erstellt man damit eine graphische Darstellung, wobei $5\%$ der ausgemessenen Werte als Messungenauigkeit
    gewählt wird. Die Rest der Tabellen sowie Abbildungen für den Schottky-Diode
     sind im Anhang zu finden. 
     \begin{figure}[H]
        \centering
        \includegraphics[width=0.8\textwidth]{figs/dioden_d1_combined.png}
        \caption{Kennlinienverlauf der Siliziumdiode in Durchlassrichtung und Sperrrichtung aus Tabellen \ref{tab:D1duchlass}, \ref{tab:D1sperr} }
        \label{dioden_d1_combined}
    \end{figure}

    \begin{figure}[H]
        \centering
        \includegraphics[width=0.8\textwidth]{figs/dioden_d2_combined.png}
        \caption{Kennlinienverlauf der Schottky-Diode in Durchlassrichtung und Sperrrichtung aus Tabellen \ref{tab:D2duchlass}, \ref{tab:D2sperr}}
        \label{dioden_d1_combined}
    \end{figure}

    
\end{aufgabe}
\newpage 
% == Aufgabe 2 ==
\begin{aufgabe}{Oszillogramm der Diodenkennlinie}
    \aufbau
    \begin{figure}[H]
        \centering
        \includegraphics[width=0.8\textwidth]{figs/Aufbau2.png}
        \caption{oberer Teil des Diodenschaltbretts\cite{anleitung}}
        \label{aufbau_2_2}
    \end{figure}
Nun soll die Messung nicht mehr analog erfolgen, sondern mittels eines Oszillographens. Hierfür wird der Strom gemäß Voraufgabe~G~(\ref{G}) in eine Spannung umgewandelt und dann im Oszillographen im x-y-Modus gegen die anliegende Spannung aufgetragen. Außerdem wird das Bild auf dem Oszilloskop zentriert, indem bei einer sehr geringen Amplitude des Signalgenerators ein Punkt entsteht, der mit den Offsets des Oszillographen auf den Ursprung verschoben wird.
    \messwerte 
    \begin{figure}[H]
        \centering
        \begin{subfigure}[b]{0.45 \textwidth}
            \includegraphics[width=\textwidth]{MesswerteVersuch2/m2_0.png}
            \caption{Silizium-Diode (D1)}
            \label{a2_0}
        \end{subfigure}
        \hfill
        \begin{subfigure}[b]{0.45 \textwidth}
            \includegraphics[width=\textwidth]{MesswerteVersuch2/m2_1.png}
            \caption{Schottky-Diode (D2)}
            \label{a2_1}
        \end{subfigure}
        \hfill
        \begin{subfigure}[b]{0.45 \textwidth}
            \includegraphics[width=\textwidth]{MesswerteVersuch2/m2_2.png}
            \caption{Zener-Diode (ZD)}
            \label{a2_2}
        \end{subfigure}
        \caption{Kennlinien der verschiedenen Dioden}
        \label{a2}
    \end{figure}
    \auswertung
    Aus den x-y-Graphen für die an der jeweiligen Diode anliegende Spannung und dem Strom, der die Diode durchfließt (bzw. einer dazu proportionalen Spannung)(\ref{a2}) lassen sich nun Diffusionsspannung und Zenerspannung (bei der Zenerdiode) graphisch ermitteln:
    \begin{align*}
        U_\mathrm{Diffusion, D1} &= \SI{0.7 +- 0.06}{\volt} \\
        U_\mathrm{Diffusion, D2} &= \SI{0.18 +- 0.06}{\volt} \\
        U_\mathrm{Diffusion, ZD} &= \SI{0.8 +- 0.24}{\volt} \\
        U_\mathrm{Zener, ZD} &= \SI{7.84 +- 0.24}{\volt} \\
    \end{align*}
    Bei den ersten beiden Dioden lässt sich keine Zenerspannung ermitteln, da im Experiment keine Spannung der passenden Größenordung verwendet wurde, um die Dioden nicht zu zerstören. Bei der Zenerdiode ist diese dann allerdings gut zu erkennen und liegt im von der Anleitung~\cite{anleitung} angegebenen Bereich von $\SIrange{3}{180}{\volt}$, der Literatur~\cite{zenerspannungen} zufolge könnte es sich um eine BZX55/C  gV2-Zenerdiode handeln, die eine Zenerspannung von $U_\mathrm{Zener}^\mathrm{theoretisch} = \SI{8.2}{\volt}$ besitzt.

    Für die Diffusionsspannung der Siliziumdiode entspricht der ermittelte Wert von $U_\mathrm{Diffusion, D1} = \SI{0.7 +- 0.06}{\volt}$ gerade dem Theoriewert $\SI{0.7}{\volt}$ der Anleitung~\cite{anleitung}. Die Diffusionsspannung der Schottkydiode soll der Theorie nach niedriger sein, was hier auch bestätigt wird: 
    \begin{equation*}
        U_\mathrm{Diffusion, D2} = \SI{0.18 +- 0.06}{\volt} < \SI{0.7 +- 0.06}{\volt} = U_\mathrm{Diffusion, D1}
    \end{equation*}
    
\end{aufgabe}
\clearpage
% == Aufgabe 3 ==
\begin{aufgabe}{Oszillogramm des Einweggleichrichters}
    \label{Aufgabe3}
    \aufbau
    \begin{figure}[H]
        \centering
        \includegraphics[width=0.8\textwidth]{figs/Aufbau3.png}
        \caption{unterer Teil des Diodenschaltbretts\cite{anleitung}}
        \label{aufbau3}
    \end{figure}
    Das Schaltbrett wird auf die Einweggleichrichtung eingestellt. Daran wird eine $\SI{50}{\hertz}$ Wechselspannung eingestellt. Als Verbraucher dient das Oszilloskop, welches die zeitliche Spannungsänderung darstellt. Zum Glätten der gleichgerichteten Spannung werden vier verschiedene Kapazitäten dem Verbraucher parallelgeschaltet.
    \messwerte
    \begin{figure}[H]
        \begin{subfigure}[b]{0.49 \textwidth}
            \includegraphics[width=\textwidth]{MesswerteVersuch2/DS0003.png}
            \caption{keine Glättung}
            \label{a3_a}
        \end{subfigure}
        \hfill
        \begin{subfigure}[b]{0.49 \textwidth}
            \includegraphics[width=\textwidth]{MesswerteVersuch2/DS0004.png}
            \caption{Glättung mit $C = \SI{2.2}{\micro\farad}$}
            \label{a3_b}
        \end{subfigure}
        \vspace{1em}
        \begin{subfigure}[b]{0.49 \textwidth}
            \includegraphics[width=\textwidth]{MesswerteVersuch2/DS0005.png}
            \caption{Glättung mit $C = \SI{22}{\micro\farad}$}
            \label{a3_c}
        \end{subfigure}
        \hfill
        \begin{subfigure}[b]{0.49 \textwidth}
            \includegraphics[width=\textwidth]{MesswerteVersuch2/DS0006.png}
            \caption{Glättung mit $C = \SI{1000}{\micro\farad}$}
            \label{a3_d}
        \end{subfigure}
        \caption{Verschiedene Glättungen eines einweggleichgerichteten Signals}
        \label{a3}
    \end{figure}
    Die mittlere Höhe der Gleichspannung und die Brumm-Amplitude werden dem Oszillographen direkt entnommen:
    \begin{table}[H]
        \centering
        \begin{tabular}{|l|l|l|l|}
            \hline
            Schaltung & Glättkapazität & mittlere Höhe der Gleichspannung & Brumm-Amplitude  \\
            \hline
            a) & $\SI{0}{\micro\farad}$ & $\SI{7.7}{\volt}$ & $\SI{24}{\volt}$ \\
            b) & $\SI{2.2}{\micro\farad}$ & $\SI{17.6}{\volt}$ & $\SI{12.6}{\volt}$ \\
            c) & $\SI{22}{\micro\farad}$ & $\SI{22.3}{\volt}$ & $\SI{2.2}{\volt}$ \\
            d) & $\SI{1000}{\micro\farad}$ & $\SI{22.4}{\volt}$ & $\SI{0.8}{\volt}$ \\
            \hline
        \end{tabular}
        \caption{Brumm- und mittlere Spannung nach Glättung}
    \end{table}
    \auswertung
    In den Grafiken~\ref{a3} kann man die Änderung des Signals für verschiedene Kapazitäten erkennen. In \ref{a3_a} ist keine Kapazität zum Glätten zwischengeschaltet, also ist das in \ref{figI} theoretisch bestimmte Bild ersichtlich. Die Brummspannung ist hier maximal und entspricht gerade der Amplitude von $\SI{24}{\volt}$. Der Mittelwert liegt deutlich unter der Amplitude, er sollte in der Theorie $\frac{\sqrt{2}}{4} U_\mathrm{Amplitude}$ entsprechen, was mit $\frac{\sqrt{2}}{4} \SI{24}{\volt} \approx \SI{8.5}{\volt}$ im Vergleich zum gemessenen Wert $\SI{7.7}{\volt}$ grob der Fall ist.

    Beim Einschalten der Kapazitäten lässt sich eine immer stärkere Glättung feststellen, einerseits in der Kurve, die immer glatter wird, andererseits in der kleiner werdenden Brumm-Amlplitude und im Steigen der mittleren Höhe der Gleichspannung.
\end{aufgabe}

\clearpage
% == Aufgabe 4 ==
\begin{aufgabe}{Oszillogramm des Zweiweggleichrichters}
    \aufbau
    \begin{figure}[H]
        \centering
        \includegraphics[width=0.8\textwidth]{figs/Aufbau4.png}
        \caption{unterer Teil des Diodenschaltbretts\cite{anleitung}}
        \label{aufbau4}
    \end{figure}
    Es wird genauso vorgegangen, wie im Einweggleichrichtungsversuchsteil, nur wird der rückläufige Gleichrichtungsweg eingeschaltet.
    \messwerte
    \begin{figure}[H]
        \begin{subfigure}[b]{0.49 \textwidth}
            \includegraphics[width=\textwidth]{MesswerteVersuch2/DS0003.png}
            \caption{keine Glättung}
            \label{a3_a}
        \end{subfigure}
        \hfill
        \begin{subfigure}[b]{0.49 \textwidth}
            \includegraphics[width=\textwidth]{MesswerteVersuch2/DS0004.png}
            \caption{Glättung mit $C = \SI{2.2}{\micro\farad}$}
            \label{a3_b}
        \end{subfigure}
        \vspace{1em}
        \begin{subfigure}[b]{0.49 \textwidth}
            \includegraphics[width=\textwidth]{MesswerteVersuch2/DS0005.png}
            \caption{Glättung mit $C = \SI{22}{\micro\farad}$}
            \label{a3_c}
        \end{subfigure}
        \hfill
        \begin{subfigure}[b]{0.49 \textwidth}
            \includegraphics[width=\textwidth]{MesswerteVersuch2/DS0006.png}
            \caption{Glättung mit $C = \SI{1000}{\micro\farad}$}
            \label{a3_d}
        \end{subfigure}
        \caption{Verschiedene Glättungen eines zweiweggleichgerichteten Signals}
    \end{figure}
    \begin{table}[H]
        \centering
        \begin{tabular}{|l|l|l|l|}
            \hline
            Schaltung & Glättkapazität & mittlere Höhe der Gleichspannung & Brumm-Amplitude  \\
            \hline
            a) & $\SI{0}{\micro\farad}$ & $\SI{15}{\volt}$ & $\SI{24}{\volt}$ \\
            b) & $\SI{2.2}{\micro\farad}$ & $\SI{21}{\volt}$ & $\SI{6.8}{\volt}$ \\
            c) & $\SI{22}{\micro\farad}$ & $\SI{23}{\volt}$ & $\SI{1.3}{\volt}$ \\
            d) & $\SI{1000}{\micro\farad}$ & $\SI{23.1}{\volt}$ & $\SI{1}{\volt}$ \\
            \hline
        \end{tabular}
        \caption{Brumm- und mittlere Spannung nach Glättung}
    \end{table}
    \auswertung
    Qualitativ verhalten sich die Glättungen gleich zu denen der Einweggleichrichtung. Im Fall ohne Glättkapazität wird allerdings die mittlere Höhe der Gleichspannung verdoppelt, da hier doppelt so viele Peaks durch die Zweigweggleichrichtung entstehen: $\frac{\SI{15}{\volt}}{\SI{7.7}{\volt}} \approx 2$. Die Brummamplitude ist für jede Kapazität innerhalb einer Messungenauigkeit von $\SI{0.5}{\volt}$ kleiner als die bei der Einweggleichrichtung, was dadurch erklärt werden kann, dass die Spannung im Fall der Zweigweggleichrichtung schon \glqq glatter\grqq\ ist. Bei der stärksten Glättung ist dieser Unterschied allerdings fast schon vernachlässigbar bzw. gar nicht vorhanden, da die Glättung insgesamt stark genug ist. 

    Zusammenfassend lässt sich sagen, dass der Zweiweggleichrichter für geringe Glättkapazitäten besser glättet, als der Einweggleichrichter, dieser Effekt aber mit zunehmenden Kapazitäten abklingt. Im Zweigwegefall sind vier Dioden nötig, bei Einweggleichrichtung reicht eine einzige, also muss für die Bevorzugung einer der Methoden die Kosten der Dioden im Verhältnis zu größeren Kapazitäten untersucht werden.

\end{aufgabe}
\clearpage
% == Aufgabe 5 ==

\begin{aufgabe}{Stabilisierung mit Zenerdiode}
    \aufbau

\begin{figure}[H]
             \centering
             \includegraphics[width=0.8\textwidth]{figs/fig2_11.png}
             \caption{Spannungsstabilisierung ohne/mit Zenerdiode \cite{anleitung}}
             \label{fig:Zenerdiode}
         \end{figure}

Dann wird das Verhalten von Spannungsstabilisierung durch eine Zenerdiode weiter untersucht. Hierzu wird die Schaltung 
\autoref{fig:Zenerdiode} aufgebaut, um eine Einweggleichrichter zu erzeugen. Die Spannung wird zusätzlich mit einem Kondensator
 mit $C= \SI{22}{\micro \farad}$ geglättet. Dieses trägt dazu bei, dass, statt der positiven Halbwelle des Sinussignals, die 
 Spannung langsamer abnimmt und es einen sägezahnförmigen Verlauf ergibt, wie in \autoref{Aufgabe3} schon betrachtet 
 und erklärt.
Der Lastwiderstand $R= \SI{150}{\ohm}$ wird anhand der Voraufgabe K gewählt, da diese im errechneten Wertebereich 
liegt(\autoref{VoraufgabeK}). Dann werden die aufgebauten Schaltungen untersucht, indem die Spannung und Stromstärke am 
Lastwiderstand $R_L$ gemessen werden, indem mit einem Potentiometer der Lastwiderstand verändert wird. Beim Aufbau ohne 
die Zenerdiode wird zusätzlich die Brummspannung aufgenommen, indem vom Oszillographen der peakpeak-Spannung abgelesen wird. 
Die angegebenen Messunsicherheiten aus \autoref{tab:Lastwiderstand_mit_Fehlern} und \autoref{tab:Stabilisierung_mit_Fehlern} werden 
basierend auf die Genauigkeit des Multimeters bestimmt. \cite{m2005-multimeter}
\messwerte
\begin{table}[h!]
\centering
\begin{tabular}{|c|c|c|c|c|c|}
\hline
\textbf{$U'$ [V]} & \textbf{$\Delta U'$ [V]} & \textbf{$I$ [$m$A]} & \textbf{$\Delta I$ [$m$A]} & \textbf{$V_\text{Brumm}$ [V]} & \textbf{$\Delta V_\text{Brumm}$ [V]} \\
\hline
0.022 & 0.001 & 44.3 & 0.5 & 1.2 & 0.1 \\
1.642 & 0.009 & 39.9 & 0.4 & 4.4 & 0.1 \\
5.120 & 0.026 & 31.9 & 0.3 & 8.8 & 0.1 \\
7.050 & 0.036 & 28.1 & 0.3 & 10.3 & 0.1 \\
9.060 & 0.046 & 24.3 & 0.3 & 10.6 & 0.1 \\
11.390 & 0.058 & 20.1 & 0.2 & 10.2 & 0.1 \\
13.980 & 0.071 & 15.4 & 0.2 & 9.2 & 0.1 \\
16.920 & 0.086 & 10.4 & 0.1 & 6.8 & 0.1 \\
19.940 & 0.101 & 5.3 & 0.1 & 4.0 & 0.1 \\
21.940 & 0.111 & 2.2 & 0.0 & 2.2 & 0.1 \\
\hline
\end{tabular}
\caption{Messwerte ohne Stabilisierung mit Lastwiderstand (aktualisierte Unsicherheiten)}
\label{tab:Lastwiderstand_mit_Fehlern}
\end{table}



\begin{table}[h!]
\centering
\begin{tabular}{|c|c|c|c|}
\hline
\textbf{$U'$ [V]} & \textbf{$\Delta U'$ [V]} & \textbf{$I$ [$m$A]} & \textbf{$\Delta I$ [$m$A]} \\
\hline
0.021 & 0.001 & 44.0 & 0.5 \\
2.456 & 0.013 & 37.6 & 0.4 \\
4.690 & 0.024 & 30.8 & 0.3 \\
5.200 & 0.027 & 27.9 & 0.3 \\
6.170 & 0.032 & 21.7 & 0.2 \\
6.980 & 0.036 & 16.0 & 0.2 \\
7.320 & 0.038 & 13.5 & 0.1 \\
7.980 & 0.041 & 7.9 & 0.1 \\
8.250 & 0.042 & 4.8 & 0.1 \\
8.410 & 0.043 & 0.8 & 0.0 \\
\hline
\end{tabular}
\caption{Messwerte mit Stabilisierung und Lastwiderstand (aktualisierte Unsicherheiten)}
\label{tab:Stabilisierung_mit_Fehlern}
\end{table}


\begin{figure}[H]
             \centering
             \includegraphics[width=0.8\textwidth]{figs/2.5.png}
             \caption{Kennlinie mit und ohne Stabilisierung einer Zenerdiode. Unsicherheiten sind eingetragen, treten im Graphen nicht auf, wenn sie zu klein zum Abbilden sind.}
             \label{fig:Stabilisierung}
         \end{figure}
         
\auswertung
Am \autoref{fig:Stabilisierung} kann man nun die Kennlinien von Widerständen mit und ohne Stabilisierung gut erkennen. Zunächst gibt es einen linearen Zusammenhang zwischen $U$ und $I$ ohne Stabilisierung. Dieses ist aus dem ohmschen Gesetz $U= R \cdot I$ zu erwarten und bestätigt die richtige Aufnahme der Daten. Mit der Zenerdiode kann man erkennen, dass sich die Spannung für kleine Ströme stabilisiert. Dies erkennt man daran, dass es einen Knick im Graphen gibt und die Spannung im Intervall $[\SI{0}{\milli\ampere}, \SI{30}{\milli\ampere}]$ zwischen $\SI{8.5}{\volt}$ und $\SI{5}{\volt}$ bleibt. Diese Stabilisierung wird durch die Zenerdiode erreicht. Sie wird in Sperrrichtung betrieben und lässt nur kleinere, vom Widerstand unabhängige Ströme hindurch, weshalb der Betrag der Spannung über einen Bereich nur wenig Veränderung aufweist. Nach dem Ohmschen Gesetz kann nun bestimmt werden, ab welcher Lastspannung die Stabilisierung erfolgt. Hierzu wird aus dem Graphen der Knickpunkt bei $U' = \SI{5 +- 1}{\volt}$ und $I = \SI{30 +- 2}{\milli \ampere}$ verwendet.

\begin{align}
    R_L &= \frac{U'}{I} = \frac{\SI{5}{\volt}}{\SI{30}{\milli \ampere}} \approx \SI{166.7}{\ohm} \\
    \Delta R_L &= \sqrt{\left( \frac{\Delta U'}{I}\right)^2 + \left( \frac{\Delta I \cdot U'}{I^2}\right)^2} \approx \SI{35.1}{\ohm}
\end{align}


Die Stabilisierung wirkt also ab einem Lastwiderstand von $R_L = \SI{166.7 +- 35.1}{\ohm}$. In der Voraufgabe K ist der Arbeitswiderstand bestimmt, damit die Stabilisierung für den Lastwiderstand von $R_L > \SI{200}{\ohm}$ erfolgt. Der berechnete Wert schließt diesen mit seiner Unsicherheit ein. Ohne die Betrachtung der Unsicherheit gibt es eine totale Abweichung von etwa $17\%$. Diese größere Abweichung entsteht, da man aus dem Graphen schwierig einen genauen Knickpunkt ablesen kann, weshalb die Unsicherheiten auch so groß gewählt werden müssen. Außerdem können die Messungen auch ungenau sein, da in der Realität systematische Fehler der Geräte und Schaltungen eine Rolle spielen.


\begin{figure}[h!]
    \centering
    % Erste Reihe
    \begin{subfigure}[b]{0.45\textwidth}
        \includegraphics[width=\textwidth]{MesswerteVersuch2/DS0014.png}
    \end{subfigure}
    \hfill
    \begin{subfigure}[b]{0.45\textwidth}
        \includegraphics[width=\textwidth]{MesswerteVersuch2/DS0017.png}
      
    \end{subfigure}

    \vspace{0.5cm} % Abstand zwischen den Zeilen

    % Zweite Reihe
    \begin{subfigure}[b]{0.45\textwidth}
        \includegraphics[width=\textwidth]{MesswerteVersuch2/DS0013.png}
          \caption{Ohne Stabilisierung}
    \end{subfigure}
    \hfill
    \begin{subfigure}[b]{0.45\textwidth}
        \includegraphics[width=\textwidth]{MesswerteVersuch2/DS0016.png}
        \caption{Mit Stabilisierung}
    \end{subfigure}


    \caption{Darstellung der Glättung und Brummspannung bei unterschiedlichen Laswiderständen.}
    \label{fig:StabBrumm}
\end{figure}

In \autoref{fig:StabBrumm} kann man den Unterschied zwischen einer Glättung und einer Glättung mit zusätzlicher Stabilisierung erkennen. Anhand der \autoref{fig:StabBrumm} und sowie der letzten Spalte \autoref{tab:Lastwiderstand_mit_Fehlern} kann man erkennen, dass die Glättung mit zunehmendem Lastwiderstand auch steigt. Dies war zu erwarten, da die Aufladezeit des Kondensators proportional zum Widerstand ist: $\tau = R \cdot C$. Bei der Stabilisierung kann man erkennen, dass diese dazu führt, dass die Brummspannung auch geringer wird. Dies führt zu einer stärkeren Glättung.

\end{aufgabe}




% === Fazit ===
\section{Fazit}
In diesem Versuch wurden Dioden und ihre Gebrauchsweisen in Stromkreisen als Gleichrichter und als Stabilisierer untersucht. 
Zunächst wurden hierfür Kennlinien einer Siliziumdiode und einer Schottky-Diode vermessen. Dabei konnte die Durchlassspannung der Siliziumdiode von $\SI{0.7 +- 0.06}{\volt}$ gemsessen werden, 
welche mit dem theoretischen Wert sehr genau übereinstimmt. Die Durchlassspannung der Schottky-Diode wurde auf $\SI{0.18 +- 0.06}{\volt}$ ermittelt, was auch gut mit dem theoretischen Wert übereinstimmte. 
Hiernach wurden Kennlinien der Silizium-, Schottky, und der Zenerdiode vermessen und auf dem Oszillosgraphen dargestellt. Diese wurden miteinander verglichen und hatten die theoretisch ermittelte Ordnung, nämlich, dass die Durchlassspannung der Siliziumdiode die der Schottkydiode übersteigt. Auch die Durchlass- und Zenerspannung sind mit den Literaturwerten im Einklang.
Dann wurde ein Gleichstrom mit unterschiedlichen Gleichrichtern hergestellt. Zunächst wurde untersucht, wie ein Kondensator die Spannung glättet und festgestellt, dass höhere Kapazitäten eine höhere Glättung bewirken. Außerdem sorgte die Zweiweggleichrichtung für ein gleichgerichteteres Signal für niedrige Kapazitäten, als die Einweggleichrichtung
Zuletzt wurde noch ein Gleichstrom mit einer Zenerdiode hergestellt und unterucht, wie dieser die Spannung stabiliert. 
Die Messergebnisse bestätigen die Spannungsstabilisierung durch die Zenerdiode im Bereich kleiner Ströme. 
Der ermittelte Lastwiderstand von \( R_L = (166{,}7 \pm 35{,}1)\,\Omega \) lag im erwarteten Bereich. 
Die Stabilisierung reduzierte zudem die Brummspannung.

% === Anhang ===%
\section{Anhang}

    
\begin{table}[h!]
    \centering
    \begin{tabular}{|l|l|}
    \hline
    \textbf{Spannung $U$ in $[V]$} & \textbf{Strom $A$ in $[mA]$} \\
    \hline
    $0.050$ & $0.02$ \\
    $0.100$ & $0.14$ \\
    $0.150$ & $0.59$ \\
    $0.200$ & $0.94$\\
    $0.247$ & $1.4$ \\
    $0.299$ & $1.8$ \\
    $0.352$ & $2.3$ \\
    $0.398$ & $2.7$\\
    $0.448$ & $3.1$ \\
    $0.502$ & $3.6$ \\
    $0.600$ & $4.5$ \\
    $2.016$ & $9.5$ \\
    $5.100$ & $45.0$ \\
    \hline
    \end{tabular}
    \caption{Kennlinie D2 in Durchlassrichtung}
    \label{tab:D2duchlass}
    \end{table}
    
    
    \begin{table}[h!]
    \centering
    \begin{tabular}{|l|l|}
    \hline
    \textbf{Spannung $U$ in $[V]$} & \textbf{Strom $A$ in $[\mu A]$} \\
    \hline
    $-1.010$ & $23$ \\
    $-2.074$ & $29$\\
    $-3.042$ & $32$\\
    $-4.03$ & $38$\\
    $-5.02$ & $43$\\
    $-6.07$ & $48$\\
    $-7.01$ & $55$\\
    $-8.01$ & $60$\\
    $-9.02$ & $65$\\
    $-10.04$ & $70$\\ 
    \hline
    \end{tabular}
    \caption{Kennlinie D2 in Sperrrichtung}
    \label{tab:D2sperr}
    \end{table}


\begin{figure}[H]
    \centering
    \includegraphics[width=0.8\textwidth]{figs/dioden_d1.png}
    \caption{Kennlinienverlauf der Siliziumdiode in Durchlassrichtung MRA4004; Sperrspannung 400V\cite{anleitung} }
    \label{dioden_d1}
\end{figure}

\begin{figure}[H]
    \centering
    \includegraphics[width=0.8\textwidth]{figs/dioden_d1_sperr.png}
    \caption{Kennlinienverlauf der Siliziumdiode in Sperrrichtung MRA4004; Sperrspannung 400V\cite{anleitung}}
    \label{dioden_d1_sperr}
\end{figure}

\begin{figure}[H]
    \centering
    \includegraphics[width=0.8\textwidth]{figs/dioden_d2.png}
    \caption{Kennlinienverlauf der Schottky-Diode in Durchlassrichtung 10BQ015; Sperrspannung 15V\cite{anleitung}}
    \label{dioden_d2}
\end{figure}

\begin{figure}[H]
    \centering
    \includegraphics[width=0.8\textwidth]{figs/dioden_d2_sperr.png}
    \caption{Kennlinienverlauf der Schottky-Diode in Durchlassrichtung 10BQ015; Sperrspannung 15V\cite{anleitung}}
    \label{dioden_d2_sperr}
\end{figure}

\clearpage
% === Abbildungsverzeichnis ===
\listoffigures
% === Literaturverzeichnis
\printbibliography

\end{document}